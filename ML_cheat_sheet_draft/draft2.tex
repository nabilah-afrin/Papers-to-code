\documentclass{article}  % Test with default article class

\usepackage{fontspec}  % For custom fonts
\usepackage{polyglossia}  % For language support

% Set English and Bangla as languages
\setdefaultlanguage{english}
\setotherlanguage{bengali}

% Set fonts for English and Bangla
\newfontfamily\bengalifont[Script=Bengali]{Kalpurush}  % You can change 'Kalpurush' to another Bangla font like 'SolaimanLipi' or 'Noto Sans Bengali'

\usepackage{amssymb,amsmath,bm}
\DeclareMathAlphabet{\mathcal}{OMS}{cmsy}{m}{n}
\usepackage{textcomp}
\newcommand\abs[1]{\left\lvert#1\right\rvert}
\usepackage{longtable}
\usepackage{algorithm2e}
\usepackage{tocbibind}
\usepackage[toc]{multitoc}
\renewcommand{\bibname}{References}
\usepackage{mathptmx}  % Times Roman as basic font
\usepackage{helvet}    % Helvetica as sans-serif font
\usepackage{courier}   % Courier as typewriter font

\usepackage{makeidx}   % Index generation
\usepackage{graphicx}  % Standard LaTeX graphics tool
\usepackage[justification=centering]{caption}
\usepackage{subfig}
\usepackage{multicol}  % For multi-column index
\usepackage{multirow}
\usepackage[bottom]{footmisc}  % Footnotes at the bottom
\usepackage[bookmarksnumbered=true,
            bookmarksopen=true,
            colorlinks=true,
            linkcolor=blue,
            anchorcolor=blue,
            citecolor=blue]{hyperref}

\graphicspath{{figures/}}

\makeindex  % For subject index generation

\begin{document}

\section{Introduction}

\section{Types of Machine Learning}
Here are the types of machine learning and their explanations in Bangla:

\begin{itemize}
    \item \textbf{Supervised Learning}
    \begin{itemize}
        \item \textbf{Classification}: {\bengalifont ডেটাকে নির্দিষ্ট ক্যাটাগরিতে ভাগ করা}
        \item \textbf{Regression}: {\bengalifont একটি কন্টিনিউয়াস রেজাল্টকে প্রেডিক্ট করা}
    \end{itemize}
    \item \textbf{Unsupervised Learning}
    \begin{itemize}
        \item \textbf{Clusters}: {\bengalifont এক রকমের ডেটা পয়েন্টগুলোকে একসাথে গ্রুপ করা}
        \item \textbf{Discovering latent factors}: {\bengalifont ডেটার মধ্যে লুকানো ফ্যাক্টর খুঁজে বের করা}
        \item \textbf{Discovering graph structure}: {\bengalifont ডেটার মধ্যে বিভিন্ন সম্পর্ক খুঁজে বের করা, যেখানে ডেটাকে নোড এবং এজ দিয়ে গ্রাফ আকারে দেখানো যায়}
        \item \textbf{Matrix completion}: {\bengalifont কোথাও ডেটা ম্যাট্রিক্সের কিছু অংশ মিসিং থাকলে, সেটা পূরণ করার চেষ্টা করা হয়}
    \end{itemize}
\end{itemize}

\end{document}
